% Objetivo del archivo: cargar y definir los parámetros básicos del paquete.
%
\usepackage[bookmarksnumbered]{hyperref}
% Permite crear enlaces externos e internos.
% Este paquete debería cargarse el último según su documentación y recomendaciones generales de la comunidad.
% Multitud de opciones y configuraciones.
%
\hypersetup{
	colorlinks=true,
	linkcolor=blue,%enlaces y referencias internas (índices, refs a figuras y secciones)
	urlcolor=cyan,
	citecolor=green,
	filecolor=magenta,
	menucolor=red,
	runcolor=brown,
}
% Tiene dos modos de uso:
%
%	A) Enlace-etiqueta: crea el enlace mostrando una etiqueta (texto) y no la dirección de enlace.
% Sintaxis de uso en texto:
%\hyperref{URL}{TEXTO}
%	{URL}: es la dirección URL (para enlace externo) o etiqueta ("label") (para enlace interno) a donde queremos que se dirija el enlace.
%	{TEXTO}: es el texto que se mostrará en el \LaTeX como enlace.
%
%	B) Enlace-directo: crea el link escribiendo la url "tal cual" (para imprimir la dirección del enlace).
% Sintaxis de uso en texto:
%\url{URL}
%	{URL}: es la dirección a la que queremos enlazar, y que se mostrará en el texto.