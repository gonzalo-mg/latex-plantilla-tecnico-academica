% Objetivo del archivo: definir la fuente de la letra; definir márgenes de página; definir espacios.
%
% Listado de unidades de espacios disponibles en latex:
%pt: punto tipográfico; 1pt=0,3515 mm.
%mm: milímetro.
%cm: centímetro.
%in: pulgada.
%ex: altura de una <<x>> en la fuente activa.
%em: ancho de una <<M>> en la fuente activa.
%mu: (em/18) de la fuente activa en modo $$.
%sp: puntos especiales; 65536sp = 1pt.
%
%COD FUENTE
% La fuente se define mediante la carga del paquete específico de cada fuente. Consultar en internet las fuentes disponibles y sus nombres en latex. Por ejemplo, {helvet} es el equivalente a Arial (que es una fuente propietaria).
\usepackage{helvet}
\renewcommand{\familydefault}{\sfdefault}% Eliminar serif.
%
%COD MÁRGENES DE PÁGINA
% Los márgenes se definen mediante la carga del paquete {geometry} que permite establecer los márgenes de cada lado de la página.
\usepackage[left=2cm,right=2cm,top=2cm,bottom=2cm]{geometry}% Multitud de opciones y configuraciones.
% Sintaxis:
%\usepackage[LADO=VALOR]{geometry}
%	LADO: lado de la hoja del que se define el margen.
%		Posibles valores: left, right, top, bottom.
%	VALOR: dimensión del margen expresada como combinación de un argumento numérico (x) y su unidad (u).
%		Sintaxis: xu
%			x: valor numérico del espacio de margen.
%			u: unidad de medida del espacio de margen.
%				Posibles valores: cm, pt.
%	geometry: nombre del paquete.
%
%COD INTERLINEADO
\linespread{1.2}
% Sintaxis:
%\linespread{x}
%	x: valor numérico, en pt.
% Como orientación: {1pt}=valor por defecto ; {1.3}=linea y media ; {1.6}=doble espacio.
%
%COD IDENTACIÓN DE PRIMERA LÍNEA DE PÁRRAFO
\setlength{\parindent}{1cm}
%COD ESPACIADO VERTICAL ENTRE PÁRRAFOS
\setlength{\parskip}{.5cm}
%COD ESPACIADO VERTICAL DE NÚMERO DE PÁGINA A FINAL DE PÁGINA
\setlength{\footskip}{3\baselineskip}
%COD NOTAS AL PIE
% Extender la línea de las notas al pie a todo el ancho de texto de la página.
\renewcommand{\footnoterule}{%
	\kern -3pt
	\hrule width \textwidth height 1pt
	\kern 2pt
}
%