% Objetivo del archivo: cargar paquetes y definir los parámetros de la gestión bibliográfica mediante biblatex.
% Las referencias bibliográficas en biblatex se almacenan en archivos de texto tipo .BIB que se definen según una estructura predeterminada y deben ser cargados al \LaTeX para poder invocar las referencias.
%
%--------------------------------------------------------------------------
%	GESTIÓN DE LA BIBLIOGRAFÍA POR BIBLATEX
%--------------------------------------------------------------------------
\usepackage[style=iso-numeric,defernumbers=true]{biblatex}% Multitud de opciones y configuraciones.
% Sintaxis:
%\usepackage[style=ESTILO]{biblatex}
%	ESTILO: valor de la opción que determina el estilo de citación a aplicar; existen varios estilos predetermiados en el paquete, a modo de ejemplo:
%		iso-numeric: estilo basado en las recomendaciones de norma ISO con clave numérica.
%		apa6: estilo basado en normas APA 6ª edición.
%
% Renombre de algunos literales al castellano:
\DefineBibliographyStrings{spanish}{
	online = {en línea}
}
%
\DefineBibliographyStrings{spanish}{
	urlalso = {url}
}
%
% Automatización de la segregación en Referencias y Bibliografía.
\DeclareBibliographyCategory{cited}
\AtEveryCitekey{\addtocategory{cited}{\thefield{entrykey}}}
%
%
%COD ARCHIVOS .BIB
%
% Carga de los archivos .BIB que conforman la base de datos bibliográfica (se admiten varios archivos):
\addbibresource{bibliografia/bibliografia.BIB}
% Recomendación para automatizar la creación de archivos .bib: https://www.mybib.com